\documentclass{article}

\usepackage{amsmath}
\usepackage[hmargin=3.5cm,vmargin=2.5cm]{geometry}

\setlength{\parskip}{8pt}
\setlength{\parindent}{0pt}

\begin{document}
 
\section*{COM2004/3004 LAB WEEK 2}

\section*{Exercise 1}

\begin{itemize}
\item Go through the MATLAB code shown in pages 4, 6, 7, 9, 10, 15, 20 of the lecture notes, i.e.,  create the dataset manually, then calculate the mean, variance, and covariance.
\item Create square matrices and calculate their determinant and inverse (see slides 31-34).
\item Solve the following simultaneous equations for $x$, $y$ and $z$.
%
\begin{eqnarray*}
-6x - 2y + 2z & = & 15 \\
3x + 4y - 3z & = & 13 \\
2x + 4y - 7z & = & -9 \\
\end{eqnarray*}

\end{itemize}

\section*{Exercise 2}

\begin{itemize}
\item Generate a sequence of first ten odd integers, ie, 1, 3, $\dots$, 19.
\item Consider the arithmetic sequence 4, 10, 16, 22, $\dots$ What is the 11th number?
\item Generate the first ten numbers of the geometric sequence  4, 12, 36, 108, $\dots$
\end{itemize}

\section*{Exercise 3}

The Taylor series 
%
\begin{equation*}
\sum_{n=0...\infty} \frac{(-1)^n}{2n+1} = 1 - \frac{1}{3} + \frac{1}{5} - \frac{1}{7} + \dots = \frac{\pi}{4}
\end{equation*}
%
can be used to calculate the value of $\pi$.

\begin{itemize}

\item Find the sums for the first 10, 100, and 1000 terms. 

\item Make the above MATLAB code into a function with the following signature:
%
\begin{verbatim}  
    function approx_pi(n)
\end{verbatim} 
%
where $n$ indicates the first $n$ terms of the Taylor series. 

\item Plot the sums up to the first 100 terms.
\end{itemize}

\section*{Exercise 4}

Write a Matlab function \texttt{choose} that calculates number of different ways of choosing $m$ items from a set of $n$ items which is given by,

\begin{equation*}
x = \frac{n!}{(n-m)!m!}
\end{equation*}


where it is assumed that $m \le n$. Can you implement your function without using the built-in function factorial. 

\section*{Exercise 5}

\begin{itemize}

\item Plot univariate normal distributions with (a) $\mu = 0$, $\sigma^2 = 1$ and with (b) $\mu = 1$, $\sigma^2 = 0.2$ (see slide 47 of the lecture notes).

\item Make the above Matlab code into a function with the following signature: 
%
\begin{verbatim} 
    function uvnpdf(mu,sigma)
\end{verbatim}

\end{itemize}


\section*{Exercise 6:}

\begin{itemize}

\item Find how to use the function \texttt{mvnrnd}
%
\begin{verbatim}
    >> help mvnrnd
\end{verbatim}

\item Using \texttt{mvnrnd}, create 1000 bivariate random data, normally distributed with the mean 
$\mu = \left( \begin{matrix} -2 \\ 3 \end{matrix}\right)$
and the covariance  $\sigma = \left( \begin{matrix} 5 & 0 \\ 0 & 5 \end{matrix}\right)$

\item Write a function \texttt{plot\_gauss} with the following signature:
%
\begin{verbatim} 
    function plot_gauss(mu,Sigma)
\end{verbatim}
%
that is able to plot 1000 data points, created with \texttt{mvnrnd}, on a plane with 
$-10 \le x \le 10$ and $-10 \le y \le 10$. 

\item Change values for $\mu$ and $\Sigma$ and plot datasets shown in pages 16-18 of this slide.

\end{itemize}

\section*{Exercise 7}

\begin{itemize}

\item Find how to use the function \texttt{mvnpdf}
%
\begin{verbatim}
    >> help mvnpdf
\end{verbatim}
%
and test the example given in help.

\item Using \texttt{mvnpdf}, write a function with the following signature:
%
\begin{verbatim}
    function draw_gauss(mu,Sigma)
\end{verbatim}
%
that is able to draw a multivariate normal distribution with given $\mu$ and $\sigma$.

\item Change values for $\mu$ and $\Sigma$ and draw bivariate normal distributions in page 55 of this slide.

\item Write your own function \texttt{my\_mvnpdf} that calculates the multivariate normal pdf. (i.e., implement the $L$-dimensional pdf in page 51)

\item Replace \texttt{mvnpdf} with \texttt{my\_mvnpdf} and test if the function \texttt{draw\_gauss} works as expected.

\end{itemize}

\end{document}